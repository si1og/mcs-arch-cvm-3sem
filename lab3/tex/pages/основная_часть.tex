\section{Цель работы}
Изучить принципы синтеза последовательностных схем на примере проектирования недвоичного вычитающего счётчика с использованием J-K триггеров и карт Карно для минимизации логических функций.

\section{Методика выполнения работы}
\subsection{Определение числа триггеров}

Количество триггеров $m$, необходимое для построения счётчика с $M$ состояниями, определяется соотношением:
\[
m \geq \lceil \log_2 M \rceil
\]

Для $M = 5$:
\[
m \geq \lceil \log_2 5 \rceil = \lceil 2.322 \rceil = 3
\]

Таким образом, для построения счётчика понадобится \textbf{3 триггера} ($Q_1$, $Q_2$, $Q_3$).

Число избыточных состояний:
\[
N = 2^m - M = 2^3 - 5 = 3
\]

\subsection{Таблица функционирования счётчика}

Вычитающий счётчик работает в последовательности: $4 \to 3 \to 2 \to 1 \to 0 \to 4 \to \ldots$

Используем естественное двоичное кодирование состояний, где $Q_1$ --- старший разряд, $Q_3$ --- младший разряд.

\begin{table}[h]
\centering
\caption{Таблица переходов вычитающего счётчика ($M = 5$)}
\begin{tabular}{|c|ccc|ccc|c|}
\hline
\multirow{2}{*}{№ сост.} & \multicolumn{3}{c|}{Состояние $t$} & \multicolumn{3}{c|}{Состояние $t+1$} & \multirow{2}{*}{Переход} \\
 & $Q_1$ & $Q_2$ & $Q_3$ & $Q_1^{t+1}$ & $Q_2^{t+1}$ & $Q_3^{t+1}$ & \\
\hline
0 & 0 & 0 & 0 & 1 & 0 & 0 & $0 \to 4$ \\
1 & 0 & 0 & 1 & 0 & 0 & 0 & $1 \to 0$ \\
2 & 0 & 1 & 0 & 0 & 0 & 1 & $2 \to 1$ \\
3 & 0 & 1 & 1 & 0 & 1 & 0 & $3 \to 2$ \\
4 & 1 & 0 & 0 & 0 & 1 & 1 & $4 \to 3$ \\
\hline
5* & 1 & 0 & 1 & --- & --- & --- & --- \\
6* & 1 & 1 & 0 & --- & --- & --- & --- \\
7* & 1 & 1 & 1 & --- & --- & --- & --- \\
\hline
\end{tabular}
\end{table}

Состояния 5, 6, 7 являются избыточными и в нормальном режиме работы не используются.

\subsection{Характеристическая таблица J-K триггера}

\begin{table}[h]
\centering
\caption{Характеристическая таблица J-K триггера}
\begin{tabular}{|c|c|c|}
\hline
Переход $Q^t \to Q^{t+1}$ & $J$ & $K$ \\
\hline
$0 \to 0$ & 0 & * \\
$0 \to 1$ & 1 & * \\
$1 \to 0$ & * & 1 \\
$1 \to 1$ & * & 0 \\
\hline
\end{tabular}
\end{table}

\subsection{Прикладные таблицы (карты Карно переходов)}

В ячейках записаны двузначные коды $Q_i^t Q_i^{t+1}$, показывающие переход триггера из текущего состояния в следующее. Прочерк <<-->> означает избыточное состояние.

\begin{center}
\begin{tabular}{|c|c|c|c|c|}
\hline
 & \multicolumn{2}{c|}{$Q_2$} & \multicolumn{2}{c|}{$\overline{Q_2}$} \\
\hline
$Q_3$ & 00 & --- & --- & 00 \\
\hline
$\overline{Q_3}$ & 00 & --- & 10 & 01 \\
\hline
 & $\overline{Q_1}$ & $Q_1$ & $Q_1$ & $\overline{Q_1}$ \\
\hline
\end{tabular}

Таблица 2. $Q_1(t) \to Q_1(t+1)$
\end{center}

\vspace{0.5cm}

\begin{center}
\begin{tabular}{|c|c|c|c|c|}
\hline
 & \multicolumn{2}{c|}{$Q_2$} & \multicolumn{2}{c|}{$\overline{Q_2}$} \\
\hline
$Q_3$ & 11 & --- & --- & 00 \\
\hline
$\overline{Q_3}$ & 10 & --- & 01 & 00 \\
\hline
 & $\overline{Q_1}$ & $Q_1$ & $Q_1$ & $\overline{Q_1}$ \\
\hline
\end{tabular}

Таблица 3. $Q_2(t) \to Q_2(t+1)$
\end{center}

\vspace{0.5cm}

\begin{center}
\begin{tabular}{|c|c|c|c|c|}
\hline
 & \multicolumn{2}{c|}{$Q_2$} & \multicolumn{2}{c|}{$\overline{Q_2}$} \\
\hline
$Q_3$ & 10 & --- & --- & 10 \\
\hline
$\overline{Q_3}$ & 01 & --- & 01 & 00 \\
\hline
 & $\overline{Q_1}$ & $Q_1$ & $Q_1$ & $\overline{Q_1}$ \\
\hline
\end{tabular}

Таблица 4. $Q_3(t) \to Q_3(t+1)$
\end{center}

\subsection{Карты Карно для входов J и K триггеров}

На основе составленных прикладных таблиц и характеристической таблицы J-K триггера составим карты Карно для входов $J$ и $K$ каждого триггера.

\subsubsection{Карты Карно для триггера $Q_1$}

\begin{center}
\begin{tabular}{|c|c|c|c|c|}
\hline
 & \multicolumn{2}{c|}{$Q_2$} & \multicolumn{2}{c|}{$\overline{Q_2}$} \\
\hline
$Q_3$ & 0 & --- & --- & 0 \\
\hline
$\overline{Q_3}$ & 0 & --- & * & 1 \\
\hline
 & $\overline{Q_1}$ & $Q_1$ & $Q_1$ & $\overline{Q_1}$ \\
\hline
\end{tabular}

Таблица 5. $J_1$
\end{center}

\begin{center}
\begin{tabular}{|c|c|c|c|c|}
\hline
 & \multicolumn{2}{c|}{$Q_2$} & \multicolumn{2}{c|}{$\overline{Q_2}$} \\
\hline
$Q_3$ & * & --- & --- & * \\
\hline
$\overline{Q_3}$ & * & --- & 1 & * \\
\hline
 & $\overline{Q_1}$ & $Q_1$ & $Q_1$ & $\overline{Q_1}$ \\
\hline
\end{tabular}

Таблица 6. $K_1$
\end{center}

\subsubsection{Карты Карно для триггера $Q_2$}

\begin{center}
\begin{tabular}{|c|c|c|c|c|}
\hline
 & \multicolumn{2}{c|}{$Q_2$} & \multicolumn{2}{c|}{$\overline{Q_2}$} \\
\hline
$Q_3$ & * & --- & --- & 0 \\
\hline
$\overline{Q_3}$ & * & --- & 1 & 0 \\
\hline
 & $\overline{Q_1}$ & $Q_1$ & $Q_1$ & $\overline{Q_1}$ \\
\hline
\end{tabular}

Таблица 7. $J_2$
\end{center}

\begin{center}
\begin{tabular}{|c|c|c|c|c|}
\hline
 & \multicolumn{2}{c|}{$Q_2$} & \multicolumn{2}{c|}{$\overline{Q_2}$} \\
\hline
$Q_3$ & 0 & --- & --- & * \\
\hline
$\overline{Q_3}$ & 1 & --- & * & * \\
\hline
 & $\overline{Q_1}$ & $Q_1$ & $Q_1$ & $\overline{Q_1}$ \\
\hline
\end{tabular}

Таблица 8. $K_2$
\end{center}

\subsubsection{Карты Карно для триггера $Q_3$}

\begin{center}
\begin{tabular}{|c|c|c|c|c|}
\hline
 & \multicolumn{2}{c|}{$Q_2$} & \multicolumn{2}{c|}{$\overline{Q_2}$} \\
\hline
$Q_3$ & * & --- & --- & * \\
\hline
$\overline{Q_3}$ & 1 & --- & 1 & 0 \\
\hline
 & $\overline{Q_1}$ & $Q_1$ & $Q_1$ & $\overline{Q_1}$ \\
\hline
\end{tabular}

Таблица 9. $J_3$
\end{center}

\begin{center}
\begin{tabular}{|c|c|c|c|c|}
\hline
 & \multicolumn{2}{c|}{$Q_2$} & \multicolumn{2}{c|}{$\overline{Q_2}$} \\
\hline
$Q_3$ & 1 & --- & --- & 1 \\
\hline
$\overline{Q_3}$ & * & --- & * & * \\
\hline
 & $\overline{Q_1}$ & $Q_1$ & $Q_1$ & $\overline{Q_1}$ \\
\hline
\end{tabular}

Таблица 10. $K_3$
\end{center}

\subsection{Минимизация и логические уравнения}

Из полученных карт Карно составляем логические уравнения для входов триггеров:

\begin{align*}
J_1 &= \overline{Q_2} \cdot \overline{Q_3} & K_1 &= 1 \\[0.3cm]
J_2 &= Q_1 & K_2 &= \overline{Q_3} \\[0.3cm]
J_3 &= Q_1 \vee Q_2 & K_3 &= 1
\end{align*}

\subsection{Описание схемы счётчика}

На основе полученных уравнений в среде Multisim была составлена схема вычитающего счётчика на трёх J-K триггерах с использованием элементов 2И и 2ИЛИ. Схема работает в цикле $4 \to 3 \to 2 \to 1 \to 0 \to 4 \to \ldots$, уменьшая значение на единицу при каждом тактовом импульсе. На рисунках представлены два последовательных состояния счётчика: состояние 4 ($Q_1 Q_2 Q_3 = 100$, рис. \ref{fig:image_2025-12-02_08-24-18.png}) и состояние 3 ($Q_1 Q_2 Q_3 = 011$, рис. \ref{fig:image_2025-12-02_08-24-44.png}) после подачи тактового импульса, что подтверждает корректность синтеза.

\screenshot{image_2025-12-02_08-24-18.png}{Состояние счётчика 4 ($Q_1 Q_2 Q_3 = 100$)}
\screenshot{image_2025-12-02_08-24-44.png}{Состояние счётчика 3 ($Q_1 Q_2 Q_3 = 011$)}

\section{Исследование микросхемы К155ИЕ6 (SN74192)}

\subsection{Описание микросхемы}

Микросхема К155ИЕ6 представляет собой реверсивный десятичный счётчик с предустановкой. Основные выводы микросхемы:

\screenshot{1.png}{Микросхема К155ИЕ6}

\begin{itemize}
    \item \textbf{A, B, C, D} — входы параллельной загрузки (предустановка значения)
    \item \textbf{QA, QB, QC, QD} — выходы счётчика
    \item \textbf{$\sim$LOAD} — вход загрузки предустановленного значения (активный низкий)
    \item \textbf{CLR} — вход сброса счётчика в 0
    \item \textbf{UP} — тактовый вход для счёта вверх
    \item \textbf{DOWN} — тактовый вход для счёта вниз
    \item \textbf{$\sim$CO} — выход переноса
    \item \textbf{$\sim$BO} — выход заёма
\end{itemize}

На входах предустановки использованы RS-триггеры (U1--U4), формирующие чистый сигнал при переключении ключей S6, S1, S2, S3. Вес каждого разряда: A=1, B=2, C=4, D=8.

\subsection{Демонстрация работы схемы}

На рисунках \ref{fig:2.png}--\ref{fig:3.png} представлена последовательность переключений, демонстрирующая работу входов предустановки микросхемы 74192N.


\screenshot{2.png}{Переключён ключ для запрета <<LOAD>>, переключены триггеры для <<DOWN>>, потом для <<DOWN>>.}

\screenshot{3.png}{Переключен триггер для <<UP>>.}

\subsection{Автоматический счёт с генератором импульсов}

Для демонстрации работы счётчика в динамическом режиме к схеме подключён генератор импульсов частотой 10 Гц и логический анализатор XLA1.

\screenshot{4.png}{Схема с генератором импульсов 10 Гц для автоматического счёта}

Генератор подключён к входу <<UP>> микросхемы К155ИЕ6, что обеспечивает автоматический счёт в режиме сложения.

\screenshot{image_2025-12-04_09-26-40.png}{Временная диаграмма работы счётчика в логическом анализаторе XLA1}

\section{Синтез счётчика с коэффициентом пересчёта 6 на основе К155ИЕ6}

Чтобы создать сётчик с модулем пересчёта 6, необходимо синтезировать схему обратной связи, сбрасывающую сётчик при достижении состояния 6 (0110 в двоичной системе).

Так как состояние 6 определяется как:

\[
Q_1Q_2Q_3Q_4 = 0110
\]

Следовательно, логическая функция, которая обнаруживает это состояние, будет выглядеть так:

\[
F = \overline{Q_1}Q_2Q_3\overline{Q_4}
\]

Эта функций принимает значение <<1>> только при комбинации сигналов, соответствующих числу 6 и сбрасывет значение сётчика (рис. \ref{fig:5.png}).

\screenshot{5.png}{Синтез схемы обратной связи для сётчика.}

\section{Результаты работы}

\begin{itemize}
  \item В ходе лабораторной работы был успешно синтезирован и исседован вычитающий счётчик с коэффициентом пересчёта 5 на J--K триггеров.
  \item С помошью карт Карно были получены минимальные логические функций возбуждения триггеров.
  \item Была изучена работа микросхемы К155ИЕ6 в разных режимах (суммирование, вычитание, задание значения).
  \item На базе данной микросхемы при помощи схемы обратной связи и дешифратора состояния <<0110>> был собран счётчик с коэффициентом пересчёта 6.
\end{itemize}

Полученные результаты подтвердили эффективность синтеза последовательностных схем и их практическую применимость.