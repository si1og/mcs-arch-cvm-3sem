\subsection*{Цель работы}
Изучить принцип работы сдвигающих регистров.

\subsection*{Краткие теоретические сведения}

Последовательные регистры предназначаются для кратковременного хранения информации, представленной в двоичном коде, и строятся на триггерах разных типов. В последовательных регистрах осуществляется логическая операция сдвига кода хранимого числа на любое число разрядов.

Сдвиг кода числа осуществляется с помощью сдвигающих импульсов, которые сдвигают все разряды кода числа с входа (сдвиг вправо) или с выхода регистра (сдвиг влево) к его выходу (входу), последовательно переводя каждый триггер регистра в состояния, соответствующее разряду кода на входе данного триггера в момент поступления очередного сдвигающего импульса.

Сдвиг вправо на $p$ разрядов соответствует операции деления на $K^p$, а сдвиг влево — операции умножения на $K^p$, где $K$ — основание системы счисления.

\subsection*{Методика выполнения работы}

В первой части работы необходимо синтезировать последовательный 4-х разрядный сдвиговый регистр. Для синтеза используются D-триггеры, так как они наиболее просты в реализации сдвиговых операций и являются синхронными.

D-триггер имеет один информационный вход $D$ и тактовый вход $C$. Принцип работы D-триггера прост: по фронту тактового импульса состояние выхода $Q$ становится равным состоянию входа $D$:
\[
Q^{t+1} = D
\]

\begin{table}[H]
\centering
\caption{Таблица истинности D-триггера}
\begin{tabular}{|c|c|}
\hline
$D$ & $Q^{t+1}$ \\
\hline
0 & 0 \\
1 & 1 \\
\hline
\end{tabular}
\end{table}

Процедура синтеза сдвигового регистра на D-триггерах значительно проще, чем на JK-триггерах. Для каждого $i$-го триггера необходимо определить его следующее состояние $Q_i^{t+1}$ в зависимости от:
\begin{itemize}
    \item Состояния соседнего триггера $Q_{i+1}^t$ (при сдвиге вправо)
    \item Состояния соседнего триггера $Q_{i-1}^t$ (при сдвиге влево)
\end{itemize}

\subsection*{Часть 1. Синтез 4-х разрядного сдвигового регистра}

Для сдвигового регистра на D-триггерах связь между триггерами определяется направлением сдвига.

\subsubsection*{Сдвиг вправо}

При сдвиге вправо каждый триггер принимает значение от соседа слева (старшего разряда):
\[
Q_i^{t+1} = Q_{i+1}^t
\]

Это означает, что вход $D_i$ каждого триггера подключается к выходу $Q_{i+1}$ предыдущего (старшего) триггера:
\[
D_i = Q_{i+1}
\]

\begin{table}[H]
\centering
\caption{Таблица переходов для сдвига вправо}
\begin{tabular}{|c|c|c|}
\hline
$Q_i^t$ & $Q_{i+1}^t$ & $Q_i^{t+1}$ \\
\hline
0 & 0 & 0 \\
0 & 1 & 1 \\
1 & 0 & 0 \\
1 & 1 & 1 \\
\hline
\end{tabular}
\end{table}

Из таблицы видно, что $Q_i^{t+1} = Q_{i+1}^t$, следовательно $D_i = Q_{i+1}$.

\subsubsection*{Сдвиг влево}

При сдвиге влево каждый триггер принимает значение от соседа справа (младшего разряда):
\[
Q_i^{t+1} = Q_{i-1}^t
\]

Вход $D_i$ каждого триггера подключается к выходу $Q_{i-1}$ следующего (младшего) триггера:
\[
D_i = Q_{i-1}
\]

\begin{table}[H]
\centering
\caption{Таблица переходов для сдвига влево}
\begin{tabular}{|c|c|c|}
\hline
$Q_i^t$ & $Q_{i-1}^t$ & $Q_i^{t+1}$ \\
\hline
0 & 0 & 0 \\
0 & 1 & 1 \\
1 & 0 & 0 \\
1 & 1 & 1 \\
\hline
\end{tabular}
\end{table}

Из таблицы видно, что $Q_i^{t+1} = Q_{i-1}^t$, следовательно $D_i = Q_{i-1}$.

\subsubsection*{Схема 4-х разрядного сдвигового регистра}

Для 4-х разрядного регистра с триггерами $Q_1, Q_2, Q_3, Q_4$ (где $Q_1$ — старший разряд, $Q_4$ — младший):

\textbf{Сдвиг вправо:}
\begin{align*}
D_1 &= D_{in} \text{ (входные данные)} \\
D_2 &= Q_1 \\
D_3 &= Q_2 \\
D_4 &= Q_3
\end{align*}

\textbf{Сдвиг влево:}
\begin{align*}
D_4 &= D_{in} \text{ (входные данные)} \\
D_3 &= Q_4 \\
D_2 &= Q_3 \\
D_1 &= Q_2
\end{align*}

\subsubsection*{Преимущества D-триггеров для сдвиговых регистров}

\begin{enumerate}
    \item \textbf{Простота синтеза} — не требуется составление карт Карно и минимизация, связь между триггерами определяется напрямую
    \item \textbf{Отсутствие дополнительной логики} — выход одного триггера подключается непосредственно к входу следующего
    \item \textbf{Синхронность} — все триггеры переключаются одновременно по фронту тактового импульса
\end{enumerate}

\screenshot{image_2025-12-11_13-05-21.png}{Схема 4-х разрядного сдвигового регистра на D-триггерах: состояние 1100 после сдвига вправо}

\screenshot{image_2025-12-11_13-04-48.png}{Схема 4-х разрядного сдвигового регистра на D-триггерах: состояние 0011 после сдвига влево}

\subsection*{Часть 2. Исследование микросхемы К155ИР13 (SN74198)}

Микросхема К155ИР13 (зарубежный аналог SN74198) представляет собой восьмиразрядный универсальный сдвиговый регистр со следующими возможностями:

\begin{itemize}
    \item Параллельная загрузка данных через входы $D1$--$D8$
    \item Последовательный сдвиг вправо через вход $DR$
    \item Последовательный сдвиг влево через вход $DL$
    \item Сброс регистра через вход $R$
\end{itemize}

Режим работы регистра определяется комбинацией управляющих входов $S0$ и $S1$:

\begin{table}[H]
\centering
\caption{Режимы работы регистра К155ИР13}
\begin{tabular}{|c|c|l|}
\hline
$S1$ & $S0$ & Режим работы \\
\hline
0 & 0 & Хранение \\
0 & 1 & Сдвиг влево \\
1 & 0 & Сдвиг вправо \\
1 & 1 & Параллельная запись \\
\hline
\end{tabular}
\end{table}

\textbf{Режим 1: Хранение (S1=0, S0=0)}

При установке управляющих входов в S1=0, S0=0 регистр сохраняет текущее состояние независимо от тактовых импульсов. Данные на входах D1-D8, DR и DL игнорируются. Это режим используется для временного удержания информации.

\textbf{Режим 2: Сдвиг влево (S1=0, S0=1)}

При установке S1=0, S0=1 регистр выполняет сдвиг влево (от младших разрядов к старшим).

\textbf{Режим 3: Сдвиг вправо (S1=1, S0=0)}

При установке S1=1, S0=0 регистр выполняет сдвиг вправо (от старших разрядов к младшим).

\textbf{Режим 4: Параллельная запись (S1=1, S0=1)}

При установке S1=1, S0=1 по тактовому импульсу происходит параллельная загрузка данных со входов D1-D8 в соответствующие разряды регистра.

\screenshot{image_2025-12-11_13-00-47.png}{Схема исследования регистра К155ИР13: состояние после сдвига вправо}


\subsection*{Часть 3. Кольцевой сдвигающий регистр}

Кольцевой регистр — это сдвиговый регистр, в котором выход последнего триггера соединён с входом первого, образуя замкнутое кольцо. При этом информация циклически сдвигается по кольцу без потерь.

Для реализации кольцевого регистра на базе К155ИР13:
\begin{itemize}
    \item Для кольцевого сдвига вправо: выход $Q8$ соединяется с входом $DR$
    \item Для кольцевого сдвига влево: выход $Q1$ соединяется с входом $DL$
\end{itemize}

\screenshot{image_2025-12-11_13-04-08.png}{Схема исследования кольцевого регистра К155ИР13: состояние после сдвига влево}

\subsection*{Результаты работы}

\begin{enumerate}
    \item Синтезирован 4-х разрядный последовательный сдвиговый регистр на базе D-триггеров. Составлены таблицы переходов для обоих направлений сдвига.
    \item Построены и проверены схемы сдвигового регистра.
    \item Изучен принцип работы универсального сдвигового регистра К155ИР13 (SN74198).
    \item Составлена схема для изучения работы универсального сдвигового регистра К155ИР13 (SN74198).
    \item Исследованы все четыре режима работы: хранение, сдвиг влево, сдвиг вправо и параллельная запись.
    \item Сконструирован кольцевой сдвигающий регистр на базе К155ИР13.
    \item Исследована работа кольцевого регистра.
\end{enumerate}