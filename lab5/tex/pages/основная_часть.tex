\section*{Цель работы}

Изучение принципов действия цифро-аналоговых и аналого-цифровых преобразователей.

\section*{Краткие теоретические сведения}

\textbf{Аналого-цифровое преобразование (АЦП)} --- операция, устанавливающая отношение двух величин: входной аналоговой $V_i$ и эталонной $V_r$. Цифровой сигнал преобразователя есть кодовое представление этого отношения. Для $n$-разрядного преобразователя число дискретных выходных уровней равно $2^n$.

\textbf{АЦП параллельного преобразования} (flash ADC) используется при требованиях к высокой скорости преобразования (видеотехника, радиолокация, цифровые осциллографы). Входной сигнал одновременно сравнивается со всеми пороговыми уровнями с помощью компараторов, смещённых по уровню опорного сигнала на один МЗР относительно друг друга. Смещение обеспечивается прецизионным резистивным делителем. Сигналы с выходов компараторов через D-триггеры подаются на приоритетный шифратор, формирующий выходной двоичный код. Скорость таких АЦП достигает 100~МГц при 8-разрядном разрешении.

\textbf{АЦП последовательного приближения} (SAR ADC) --- наиболее распространённый тип АЦП со средним и высоким быстродействием. Метод основан на аппроксимации входного сигнала двоичным кодом с последовательной проверкой правильности для каждого разряда. Преобразование начинается с установки СЗР в единицу (оценка половины диапазона), затем компаратор сравнивает выход ЦАП с входным сигналом, определяя, сохранить или сбросить разряд. Процесс повторяется для всех разрядов, что даёт выходной сигнал за $n$ тактов для $n$-разрядного преобразователя.

\section*{Ход работы}

\subsection*{Часть 1. Синтез АЦП параллельного преобразования}

Для синтеза 2-разрядного АЦП параллельного преобразования определим количество уровней квантования. Для 2-разрядного преобразователя число уровней равно $2^2 = 4$, что требует 3 компараторов (для разделения на 4 диапазона).

Пусть диапазон входного напряжения составляет от 0 до 10~В. Тогда шаг квантования:
\begin{equation}
    \Delta V = \frac{10}{4} = 2.5~\text{В}
\end{equation}

Пороговые уровни компараторов устанавливаются на 2.5~В, 5~В и 7.5~В соответственно. Резистивный делитель формируется из резисторов с соотношением сопротивлений $1:2:2:1$ (1~кОм, 2~кОм, 2~кОм, 1~кОм от земли к источнику 10~В).

\textbf{Таблица истинности приоритетного шифратора:}

Обозначим выходы компараторов как $x$, $y$, $z$ (от верхнего к нижнему порогу), а выходные биты как $f_1$ (старший) и $f_2$ (младший).

\begin{center}
\begin{tabular}{ccc|cc|l}
\hline
$x$ & $y$ & $z$ & $f_1$ & $f_2$ & Диапазон входа \\
\hline
0 & 0 & 0 & 0 & 0 & $V_{in} < 2.5$~В \\
0 & 0 & 1 & 0 & 1 & $2.5 \leq V_{in} < 5$~В \\
0 & 1 & 1 & 1 & 0 & $5 \leq V_{in} < 7.5$~В \\
1 & 1 & 1 & 1 & 1 & $V_{in} \geq 7.5$~В \\
\hline
\end{tabular}
\end{center}

\textbf{Синтез логических функций:}

По таблице истинности получаем:
\begin{align}
    f_1 &= \bar{x}yz \lor xyz = yz \\
    f_2 &= \bar{x}\bar{y}z \lor xyz = z(\bar{x}\bar{y} \lor xy) = z(x \leftrightarrow y)
\end{align}

Функция $f_1$ реализуется элементом AND2, функция $f_2$ --- элементами XNOR2 и AND2.

\textbf{Структурная схема АЦП:}

Схема включает:
\begin{itemize}
    \item Резистивный делитель напряжения (R1=1~кОм, R2=2~кОм, R3=2~кОм, R4=1~кОм)
    \item Источник опорного напряжения 10~В
    \item 3 компаратора (COMPARATOR\_IDEAL)
    \item 3 D-триггера для синхронизации
    \item Генератор тактовых импульсов 50~Гц
    \item Комбинационную логику (XNOR2, AND2)
    \item Индикаторы выходов F1, F2
\end{itemize}

Входной сигнал подаётся через источник напряжения V3 (управляемый клавишей A) на инвертирующие входы всех компараторов. На неинвертирующие входы подаются пороговые напряжения с делителя: 7.5~В, 5~В и 2.5~В.

\screenshot{image_2025-12-11_13-36-09.png}{Схема АЦП параллельного преобразования в Multisim}

\subsection*{Часть 2. Синтез АЦП последовательного приближения}

Для 2-разрядного АЦП последовательного приближения структура включает:
\begin{itemize}
    \item 2-разрядный ЦАП (R-2R матрица)
    \item Компаратор для сравнения выхода ЦАП с входным сигналом
    \item Регистр последовательного приближения (SAR)
    \item Контроллер последовательности
\end{itemize}

\textbf{Алгоритм работы:}

\begin{enumerate}
    \item Установка СЗР в 1: код $10_2$, ЦАП выдаёт $V_{ref}/2 = 5$~В
    \item Сравнение: если $V_{in} \geq 5$~В, СЗР остаётся 1, иначе сбрасывается в 0
    \item Установка МЗР в 1 и аналогичное сравнение
    \item После 2 тактов получаем 2-битный код
\end{enumerate}

Для диапазона 0--10~В соответствие кодов:
\begin{center}
\begin{tabular}{c|c}
\hline
Код & Диапазон напряжения \\
\hline
00 & 0 -- 2.5~В \\
01 & 2.5 -- 5~В \\
10 & 5 -- 7.5~В \\
11 & 7.5 -- 10~В \\
\hline
\end{tabular}
\end{center}

\screenshot{image_2025-12-11_14-32-04.png}{Схема АЦП последовательного приближения в Multisim}
\section*{Результаты работы}

В результате выполнения работы:

\begin{enumerate}
    \item Синтезирован 2-разрядный АЦП параллельного преобразования на основе трёх компараторов и приоритетного шифратора.
    \item Синтезирован 2-разрядный АЦП последовательного приближения с использованием резистивного делителя, компараторов и управляющей логики.
\end{enumerate}