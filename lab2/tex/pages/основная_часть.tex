\subsection*{Цель работы}

Синтезировать и построить схемы полусумматора и одноразрядного сумматора в соответствии с данными таблицами истинности. Изучить принципы работы суммирующего устройства (АЛУ К155ИП3).

\subsection*{Методика выполнения работы}

В соответствии с таблицей \ref{tab:one_sum} были составлены логические выражения для одноразрядного сумматора.

\[
\begin{aligned}
S &= \bar{x}\bar{y}z + \bar{x}y\bar{z} + x\bar{y}\bar{z} + xyz
   = z(\bar{x}\bar{y} + xy) + \bar{z}(\bar{x}y + x\bar{y}) \\
   &= z(x \oplus y) + \bar{z}(\overline{x \oplus y})
   = x \oplus y \oplus z, \\[6pt]
P &= \bar{x}yz + x\bar{y}z + xy\bar{z} + xyz
   = yz(\bar{x} + x) + x(\bar{y}z + y\bar{z})
   = yz + x(y \oplus z).
\end{aligned}
\]

\begin{table}[h!]
\centering
\caption{Таблица истинности однoразрядного сумматора}
\begin{tabularx}{\textwidth}{|X|X|X|X|X|}
\hline
$X$ & $Y$ & $Z$ & $S$ & $P$ \\ \hline
0 & 0 & 0 & 0 & 0 \\ \hline
0 & 0 & 1 & 1 & 0 \\ \hline
0 & 1 & 0 & 1 & 0 \\ \hline
0 & 1 & 1 & 0 & 1 \\ \hline
1 & 0 & 0 & 1 & 0 \\ \hline
1 & 0 & 1 & 0 & 1 \\ \hline
1 & 1 & 0 & 0 & 1 \\ \hline
1 & 1 & 1 & 1 & 1 \\ \hline
\end{tabularx}
\label{tab:one_sum}
\end{table}

Далее в соответствии с таблицей \ref{tab:half_sum} были сосоставлены логические выражения для полусумматора.

\[
\begin{aligned}
S &= \bar{x}y + x\bar{y} = x \oplus y, \\[4pt]
P &= xy.
\end{aligned}
\]

\begin{table}[h!]
\centering
\caption{Таблица истинности для полусумматора}
\begin{tabularx}{\textwidth}{|X|X|X|X|}
\hline
$X$ & $Y$ & $S$ & $P$ \\ \hline
0 & 0 & 0 & 0 \\ \hline
0 & 1 & 1 & 0 \\ \hline
1 & 0 & 1 & 0 \\ \hline
1 & 1 & 0 & 1 \\ \hline
\end{tabularx}
\label{tab:half_sum}
\end{table}

\par По полученным логическим выражениям были построены схемы полусумматора и одноразрядного сумматора.
В схемах использовались следующие элементы: источник питания напряжением 5~В, резистор номиналом 1~кОм,
переключатели (ключи), индикаторные лампочки, а также логические элементы \textbf{XOR}, \textbf{AND} и \textbf{OR}.

\subsection*{Протокольная часть работы}

В первой части работы были проверены схемы полусумматора и одноразрядного сумматора. В соответствии с функциями, синтезированными из таблиц истинности \ref{tab:one_sum} и \ref{tab:half_sum}.

Для проверки корректности работы схем были использованы лампочки. Вкл. - 1, Выкл. - 0. Исходные переменные задавались переключателями (ключами).

На рисунках \ref{fig:sum2.PNG} и \ref{fig:sum1.PNG} представлены полученные схемы 3-разрядного сумматора и полусумматора соответственно.

\screenshot{sum2.PNG}{Схема 3-разрядного сумматора}
\screenshot{sum1.PNG}{Схема полусумматора}

Далее, во второй части работы, была построена схема с использованием микросхемы АЛУ К155ИП3 (рис. \ref{fig:ALU.PNG}) и изучены принципы работы этой микросхемы. Данная микросхема позволяет выполнять не только арифметические операции, но еще и логические.

\begin{itemize}
    \item M -- вход, задающий режим работы АЛУ: при $M = 0$ происходит работа в режиме арифметических операций, при $M = 1$ -- в режиме логических;

    \item Cn -- вход переноса, используемый в случаях, когда производятся операции над числами с разрядностью больше четырех;

    \item A0--A3 -- информационные входы первого 4-х разрядного операнда (A);

    \item B0--B3 -- информационные входы второго 4-х разрядного операнда (B);

    \item S0--S3 -- входы выбора операции (4 бита управления);

    \item F0--F3 -- выходы, по которым задается выполняемая операция;

    \item Cn+4 -- выход переноса (для каскадирования микросхем);

    \item P, G -- выходы для организации быстрого переноса;

    \item A=B -- выход сравнения (активен при равенстве операндов).
\end{itemize}

Микросхема выполняет 16 различных арифметических операций (сложение, вычитание, инкремент, декремент и др.) при $M = 0$ и 16 логических операций (AND, OR, XOR, NOT и др.) при $M = 1$. Выбор конкретной операции осуществляется комбинацией сигналов на входах S0--S3.

\screenshot{ALU.PNG}{Схема АЛУ на микросхеме К155ИП3}


На схеме (рис. \ref{fig:ALU.PNG}) входные переключатели S1 формируют входные операнды и управляющие сигналы для микросхемы АЛУ К155ИП3.

Согласно положению переключателей S1, на входы микросхемы подаются входные операнды A и B, а также управляющие сигналы S0--S3, определяющие выполняемую операцию. Вход M установлен в состояние, соответствующее арифметическому режиму работы, что позволяет микросхеме выполнять арифметические операции над операндами. Дополнительно задается значение входа переноса Cn. При данной комбинации входных сигналов микросхема К155ИП3 выполняет заданную арифметическую операцию, результат которой отображается на светодиодах X1--X4, показывающих выходные биты F0--F3. Светодиод X4 индицирует состояние выхода переноса Cn+4, который используется при каскадировании микросхем для обработки многоразрядных чисел. Инвертор U2 обеспечивает правильную логику индикации одного из выходных сигналов.
\subsection*{Результаты работы}

В ходе выполнения лабораторной работы были синтезированы и построены схемы (рис. \ref{fig:sum2.PNG} и \ref{fig:sum1.PNG}) полусумматора и одноразрядного сумматора на основе таблиц истинности. Получены логические выражения для выходных функций $S$ и $P$ обеих схем. Схемы были реализованы с использованием логических элементов XOR, AND и OR, и их корректность была проверена экспериментально с помощью переключателей и индикаторных лампочек.

Во второй части работы была изучена микросхема АЛУ К155ИП3 и построена схема, демонстрирующая её работу (рис. \ref{fig:ALU.PNG}). Изучены назначение входов и выходов микросхемы, принципы выбора режима работы (арифметический или логический) и управления операциями. Экспериментально подтверждена возможность выполнения микросхемой различных арифметических и логических операций над 4-разрядными операндами в зависимости от комбинации управляющих сигналов.
