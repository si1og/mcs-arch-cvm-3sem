\subsection*{Цель работы}

Синтезировать и построить схемы полусумматора и одноразрядного сумматора в соответствии с данными таблицами истинности. Изучить принципы работы суммирующего устройства (АЛУ К155ИП3).

\subsection*{Методика выполнения работы}

В соответствии с таблицей \ref{tab:one_sum} были составлены логические выражения для одноразрядного сумматора.

\[
\begin{aligned}
S &= \bar{x}\bar{y}z + \bar{x}y\bar{z} + x\bar{y}\bar{z} + xyz
   = z(\bar{x}\bar{y} + xy) + \bar{z}(\bar{x}y + x\bar{y}) \\
   &= z(x \oplus y) + \bar{z}(\overline{x \oplus y})
   = x \oplus y \oplus z, \\[6pt]
P &= \bar{x}yz + x\bar{y}z + xy\bar{z} + xyz
   = yz(\bar{x} + x) + x(\bar{y}z + y\bar{z})
   = yz + x(y \oplus z).
\end{aligned}
\]

\begin{table}[h!]
\centering
\caption{Таблица истинности однoразрядного сумматора}
\begin{tabularx}{\textwidth}{|X|X|X|X|X|}
\hline
$X$ & $Y$ & $Z$ & $S$ & $P$ \\ \hline
0 & 0 & 0 & 0 & 0 \\ \hline
0 & 0 & 1 & 1 & 0 \\ \hline
0 & 1 & 0 & 1 & 0 \\ \hline
0 & 1 & 1 & 0 & 1 \\ \hline
1 & 0 & 0 & 1 & 0 \\ \hline
1 & 0 & 1 & 0 & 1 \\ \hline
1 & 1 & 0 & 0 & 1 \\ \hline
1 & 1 & 1 & 1 & 1 \\ \hline
\end{tabularx}
\label{tab:one_sum}
\end{table}

Далее в соответствии с таблицей \ref{tab:half_sum} были сосоставлены логические выражения для полусумматора.

\[
\begin{aligned}
S &= \bar{x}y + x\bar{y} = x \oplus y, \\[4pt]
P &= xy.
\end{aligned}
\]

\begin{table}[h!]
\centering
\caption{Таблица истинности для полусумматора}
\begin{tabularx}{\textwidth}{|X|X|X|X|}
\hline
$X$ & $Y$ & $S$ & $P$ \\ \hline
0 & 0 & 0 & 0 \\ \hline
0 & 1 & 1 & 0 \\ \hline
1 & 0 & 1 & 0 \\ \hline
1 & 1 & 0 & 1 \\ \hline
\end{tabularx}
\label{tab:half_sum}
\end{table}

\par По полученным логическим выражениям были построены схемы полусумматора и одноразрядного сумматора.
В схемах использовались следующие элементы: источник питания напряжением 5~В, резистор номиналом 1~кОм,
переключатели (ключи), индикаторные лампочки, а также логические элементы \textbf{XOR}, \textbf{AND} и \textbf{OR}.

\subsection*{Протокольная часть работы}

В первой части работы были проверены схемы полусумматора и одноразрядного сумматора. В соответствии с функциями, синтезированными из таблиц истинности \ref{tab:one_sum} и \ref{tab:half_sum}.

Для проверки корректности работы схем были использованы лампочки. Вкл. - 1, Выкл. - 0. Исходные переменные задавались переключателями (ключами).

На рисунках \ref{fig:sum2.PNG} и \ref{fig:sum1.PNG} представлены полученные схемы 3-разрядного сумматора и полусумматора соответственно.

\screenshot{sum2.PNG}{Схема 3-разрядного сумматора}
\screenshot{sum1.PNG}{Схема полусумматора}

Далее, во второй части работы, была построена схема с использованием микросхемы АЛУ К155ИП3, которая позволяет упростить реализацию дешифратора. Графическая схема представлена на рис.~\ref{fig:img2.PNG}. Данная микросхема представляет собой
универсальное устройство, которое может использоваться как двойной
дешифратор 2 на 4, как дешифратор 3 на 8, а также в качестве демультиплексора
1 на 4 или 1 на 8.

\screenshot{ALU.PNG}{Схема АЛУ на микросхеме К155ИП3}


На схеме (рис.~\ref{fig:img2.PNG}) входные переключатели $S3$ и $S4$ формируют адресные сигналы $B$ и $A$, а ключи $S2$, $S6$ задают разрешение групп $EA$ и $EB$.
Информационные входы $DA$ и $DB$ установлены в нулевой уровень, поэтому микросхема работает в режиме дешифратора.
При выбранной комбинации адресных сигналов активным становится один выход из группы: он переходит в логический ноль, вследствие чего загорается соответствующий индикатор.
Таким образом, светодиоды показывают, какая именно линия дешифратора выбрана при текущем адресе.

При подаче информационного сигнала на вход $DA$ (вывод~15) или $DB$
(вывод~1) микросхема работает в режиме демультиплексора: сигнал передаётся
на один из выходов, выбранный кодом на адресных входах. Если объединить
входы $DA$ и $DB$, то устройство функционирует как полный дешифратор 3 на 8,
где третий адресный разряд задаётся этим объединённым входом (с весом $2^2=4$).

В рамках работы была собрана схема включения микросхемы и исследованы
её режимы функционирования, подтердилось, что микросхема что К155ИД4 корректно выполняет дешифрацию и демультиплексирование в соот. с логикой своей работы.

\subsection*{Результаты работы}

В ходе выполнения лабораторной работы был синтезирован и исследован трёхразрядный дешифратор на основе таблицы истинности.
В среде <<Multisim>> была реализована модель устройства, показавшая корректность работы схемы: при каждой комбинации входных сигналов активируется только один выход, соответствующий поданному двоичному коду.

Также был подробно изучен принцип функционирования микросхемы К155ИД4 (SN74155).
Были рассмотрены её основные режимы работы: два дешифратора $2 \times 4$, один дешифратор $3 \times 8$, а также демультиплексоры $1 \times 4$ и $1 \times 8$.
Составлена схема включения микросхемы, реализованная в среде <<Multisim>>, и проведено моделирование при различных комбинациях входных сигналов.

Результаты моделирования подтвердили правильность функционирования как дешифратора, так и демультиплексора: устройство корректно реагирует на изменения адресных входов и формирует активный сигнал на соответствующем выходе в соответствии с логикой своей работы.
