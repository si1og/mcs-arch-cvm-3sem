\subsection*{Цель работы}

Собрать 3-разрядный дешифратор, а также исследлвать принцип его работы.  Изучить архитеркуру и режимы работы микросхемы К155ИД4 (дешифратор 2 на 4, дешифратор 3 на 4 и демультиплексор), используя среду <<Multisim>>.

\subsection*{Методика выполнения работы}

На основании таблицы~1 необходимо составить восемь логических выражений (см. примечание~1), которые легли в основу построения схемы 3-разрядного дешифратора.  
Для реализации схемы следует использовать следующие элементы: источник питания, заземление, резистор на 1 кОм, переключатели, инверторы, блок логического <<И>> и индикаторы. В процессе проверки схемы было подтверждено, что дешифратор корректно передаёт сигнал на выбранный выход.

После с среде <<Multisim>> необходимо реализовать моделт дкшифратора: задась входные сигналы переключателями, а выходные -- через логические элменты.

Во второй части работы необходио изучить принципы функционирования дешифратора К155ИД4. При этом, помимо указанных компонентов, пользоваться микросхемным элементом 74LS155N в среде <<Multisim>>.

\begin{table}[h!]
\centering
\caption{Таблица истинности для 3-разрядного дешифратора}
\begin{tabularx}{\textwidth}{|c|X|X|X|X|}
\hline
№ набора & $x_{1}$ & $x_{2}$ & $x_{3}$ & $y$ \\ \hline
0 & 0 & 0 & 0 & 1 \\ \hline
1 & 0 & 0 & 1 & 0 \\ \hline
2 & 0 & 1 & 0 & 0 \\ \hline
3 & 0 & 1 & 1 & 1 \\ \hline
4 & 1 & 0 & 0 & 1 \\ \hline
5 & 1 & 0 & 1 & 1 \\ \hline
6 & 1 & 1 & 0 & 0 \\ \hline
7 & 1 & 1 & 1 & 1 \\ \hline
\end{tabularx}
\end{table}

\noindent\textbf{Примечание 1:} Логические выражения для выходов дешифратора:  

\begin{flalign*}
0.&\;\; \bar{x}_2 \cdot \bar{x}_1 \cdot \bar{x}_0 = y_0 &\\
1.&\;\; \bar{x}_2 \cdot \bar{x}_1 \cdot x_0 = y_1 &\\
2.&\;\; \bar{x}_2 \cdot x_1 \cdot \bar{x}_0 = y_2 &\\
3.&\;\; \bar{x}_2 \cdot x_1 \cdot x_0 = y_3 &\\
4.&\;\; x_2 \cdot \bar{x}_1 \cdot \bar{x}_0 = y_4 &\\
5.&\;\; x_2 \cdot \bar{x}_1 \cdot x_0 = y_5 &\\
6.&\;\; x_2 \cdot x_1 \cdot \bar{x}_0 = y_6 &\\
7.&\;\; x_2 \cdot x_1 \cdot x_0 = y_7 &
\end{flalign*}

\subsection*{Синтез схемы дешифратора}

В соответствии с таблицей~2 был синтезирован 3-разряддый дешифратор.  
Дешифратор представляет собой логическую схему, преобразующую двоичный код, поступающий на её входы, в управляющий сигнал, который появляется только на одном выходе, номер которого совпадает с кодом на входе.
\begin{table}[H]
\centering
\caption{Таблица истинности дешифратора 3-разрядного числа}
\begin{tabularx}{\textwidth}{|>{\centering\arraybackslash}X|
    *{11}{>{\centering\arraybackslash}p{0.05\textwidth}|}}
\hline
№ & $x_{2}$ & $x_{1}$ & $x_{0}$ & $y_{0}$ & $y_{1}$ & $y_{2}$ & $y_{3}$ & $y_{4}$ & $y_{5}$ & $y_{6}$ & $y_{7}$ \\ \hline
0 & 0 & 0 & 0 & 1 & 0 & 0 & 0 & 0 & 0 & 0 & 0 \\ \hline
1 & 0 & 0 & 1 & 0 & 1 & 0 & 0 & 0 & 0 & 0 & 0 \\ \hline
2 & 0 & 1 & 0 & 0 & 0 & 1 & 0 & 0 & 0 & 0 & 0 \\ \hline
3 & 0 & 1 & 1 & 0 & 0 & 0 & 1 & 0 & 0 & 0 & 0 \\ \hline
4 & 1 & 0 & 0 & 0 & 0 & 0 & 0 & 1 & 0 & 0 & 0 \\ \hline
5 & 1 & 0 & 1 & 0 & 0 & 0 & 0 & 0 & 1 & 0 & 0 \\ \hline
6 & 1 & 1 & 0 & 0 & 0 & 0 & 0 & 0 & 0 & 1 & 0 \\ \hline
7 & 1 & 1 & 1 & 0 & 0 & 0 & 0 & 0 & 0 & 0 & 1 \\ \hline
\end{tabularx}
\end{table}

На основе полученной таблицы истинности была составлена схема дешифратора, собранная из базовых логических элементов. Для реализации использовались три входных ключа, инверторы, элементы И, а также блоки индикации, показывающие активный выход. Схема представлена на рис.~\ref{fig:img1.PNG}.

\screenshot{img1.PNG}{Схема 3-разрядного дешифратора на логических элементах}

Во второй части работы после проверки корректности работы схемы и анализа её работы была построена схема с использованием микросхемы К155ИД4, которая позволяет упростить реализацию дешифратора. Графическая схема представлена на рис.~\ref{fig:img2.PNG}. Данная микросхема представляет собой
универсальное устройство, которое может использоваться как двойной
дешифратор 2 на 4, как дешифратор 3 на 8, а также в качестве демультиплексора
1 на 4 или 1 на 8.

\screenshot{img2.PNG}{Схема дешифратора на микросх  еме К155ИД4 (SN74155)}


На схеме (рис.~\ref{fig:img2.PNG}) входные переключатели $S3$ и $S4$ формируют адресные сигналы $B$ и $A$, а ключи $S2$, $S6$ задают разрешение групп $EA$ и $EB$.  
Информационные входы $DA$ и $DB$ установлены в нулевой уровень, поэтому микросхема работает в режиме дешифратора.  
При выбранной комбинации адресных сигналов активным становится один выход из группы: он переходит в логический ноль, вследствие чего загорается соответствующий индикатор.  
Таким образом, светодиоды показывают, какая именно линия дешифратора выбрана при текущем адресе.  

При подаче информационного сигнала на вход $DA$ (вывод~15) или $DB$
(вывод~1) микросхема работает в режиме демультиплексора: сигнал передаётся
на один из выходов, выбранный кодом на адресных входах. Если объединить
входы $DA$ и $DB$, то устройство функционирует как полный дешифратор 3 на 8,
где третий адресный разряд задаётся этим объединённым входом (с весом $2^2=4$).

В рамках работы была собрана схема включения микросхемы и исследованы
её режимы функционирования, подтердилось, что микросхема что К155ИД4 корректно выполняет дешифрацию и демультиплексирование в соот. с логикой своей работы.

\subsection*{Результаты работы}

В ходе выполнения лабораторной работы был синтезирован и исследован трёхразрядный дешифратор на основе таблицы истинности.  
В среде <<Multisim>> была реализована модель устройства, показавшая корректность работы схемы: при каждой комбинации входных сигналов активируется только один выход, соответствующий поданному двоичному коду.

Также был подробно изучен принцип функционирования микросхемы К155ИД4 (SN74155).  
Были рассмотрены её основные режимы работы: два дешифратора $2 \times 4$, один дешифратор $3 \times 8$, а также демультиплексоры $1 \times 4$ и $1 \times 8$.  
Составлена схема включения микросхемы, реализованная в среде <<Multisim>>, и проведено моделирование при различных комбинациях входных сигналов.

Результаты моделирования подтвердили правильность функционирования как дешифратора, так и демультиплексора: устройство корректно реагирует на изменения адресных входов и формирует активный сигнал на соответствующем выходе в соответствии с логикой своей работы.
